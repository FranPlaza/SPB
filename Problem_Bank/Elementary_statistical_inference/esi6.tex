%@ Subject: Elementary Statistical Inference

\question[30] Se sabe que la duraci\'on, en horas, de un foco de 75 watts tiene una distribuci\'on aproximadamente normal con desviaci\'on est\'andar poblacional de 25 horas. Se toma una muestra aleatoria de 16 focos, la cual resulta tener una duraci\'on media de 1014 horas. \noaddpoints \begin{parts} \part[10] Confeccione un intervalo de confianza del 95$\%$ para la duraci\'on media. \part[10] ?`Es posible suponer que la duraci\'on media difiere de las 1100 horas, indicadas por el fabricante? Utilice un nivel de significancia del 5\%. Justifique. \part[10] ?`Qu\'e tama\~no muestral se requiere para estimar la duraci\'on media de los focos, con una confianza del 98.12\% y una amplitud de 4 horas? \end{parts}

\begin{solution}
Sea $X: \{$ Duranci\'on, en horas, de un foco de 75 watts  $\}$. Por enunciado $X \sim N(\mu,25^2)$. El intervalo de confianza pedido est\'a dado por:\begin{align*}\left[ \overline{X} \mp Z_{1-\alpha /2}\dfrac{\sigma}{\sqrt{n}} \right] \end{align*}En donde $\alpha=0.05$ y $Z_{1-\alpha /2}=1.96$. Luego, reemplazando con la informaci\'on de enunciado, se tiene que el intervalo pedido es: \begin{align*}IC_{95\%}(\mu)=\left[1001.8;1026.3\right]\end{align*} Por enunciado, se tiene la siguiente prueba de hip\'otesis: \begin{align*}H_0:& \mu = 1100 \\H_1:& \mu \neq 1100\end{align*}Por lo que, utilizando el estad\'istico de prueba $Z$ rechazaremos la hip\'otesis nula si: \begin{align*}Z\leq -1,96 \hspace{20pt} \text{o si} \hspace{20pt}  Z\geq 1,96\end{align*} Luego, el valor de nuestro estad\'istico est\'a dado por: $Z=-13.76$. Utilizando el intervalo de confianza del item a), y dado que la amplitud es 4 horas. Se tiene: \begin{align*}2Z_{1-\alpha /2}\dfrac{\sigma}{\sqrt{n}}=4 \end{align*} En donde $\alpha=0.0188$ y $Z_{1-\alpha /2}=2.35$. Luego, despejando para $n$ y usando el hecho que $\sigma=25$, se tiene que el tama\~no de muestra pedido es: $n=862,89 \approx 863$ 
\end{solution}

